%--------------------------------------------------------------------------
%	PACKAGES AND OTHER DOCUMENT CONFIGURATIONS
%--------------------------------------------------------------------------
\documentclass[12pt,a4paper]{article}
\usepackage[utf8]{inputenc}
\usepackage[english]{babel}
\usepackage[T1]{fontenc}
\usepackage{amsmath}
\usepackage{amsfonts}
\usepackage{amssymb}
\usepackage{graphicx}
\usepackage{lmodern}
\usepackage[left=3cm,right=3cm,top=2.5cm,bottom=2.5cm]{geometry}

\usepackage{fancyhdr} % Required for custom headers
\usepackage{lastpage} % Required to determine the last page for the footer
\usepackage{extramarks} % Required for headers and footers
\usepackage[usenames,dvipsnames]{color} % Required for custom colors
\usepackage{graphicx} % Required to insert images
\usepackage{caption}
\usepackage{subcaption}
\usepackage{listings} % Required for insertion of code
%\usepackage{courier} % Required for the courier font
\usepackage{verbatim}
\usepackage{multirow}
\usepackage{eurosym}
\usepackage{url}
\usepackage{hyperref}

\lstset{
	tabsize=4,
	rulecolor=,
    basicstyle=\scriptsize,
    upquote=true,
    aboveskip={1.5\baselineskip},
    columns=fixed,
    showstringspaces=false,
    extendedchars=true,
    breaklines=true,
    %prebreak = \raisebox{0ex}[0ex][0ex]{\ensuremath{\hookleftarrow}},
    frame=single,
    showtabs=false,
    showspaces=false,
    showstringspaces=false,
    numbers=left
}

\setlength\parindent{0pt} % Removes all indentation from paragraphs

% sections in Alph
% \renewcommand{\thesection}{\Alph{section}}

\begin{document}
	
%--------------------------------------------------------------------------
%	TITLE PAGE
%--------------------------------------------------------------------------
\begin{titlepage}
\newcommand{\HRule}{\rule{\linewidth}{0.5mm}} % Defines a new command for the horizontal lines, change thickness here
\centering % Center everything on the page
 
%	HEADING SECTIONS
\begin{figure}[!h]
	\begin{center}
	\includegraphics[height=4cm]{liu.jpg}
	\end{center}
\end{figure}

%\null
%\vspace{1cm}
%\textsc{\Large Linköping University}\\[1cm] % Name of your university/college
\textsc{\large Department of Computer and Information Science}\\[0.5cm] % Major heading such as course name
\textsc{\large TDTS06 Computer Networks}\\[2.5cm] % Minor heading such as course title

%	TITLE SECTION

\HRule \\[0.4cm]
{ \LARGE \bfseries Net Ninny: A Web Proxy Based Service\\[0.4cm] % Title of your document
\Large \bfseries Report} \\[0.4cm]

\HRule \\[1.5cm]


%	AUTHOR SECTION

\large
\begin{centering}
Group C/D 13 \\[0.2cm]
\end{centering}
{\begin{tabular}{lll}
\textsc{Chvátal} & Martin & march011@student.liu.se\\
\textsc{Peschke} & Lena & lenpe782@student.liu.se\\
\end{tabular}}
\\[1.5cm]

\normalsize
Teaching assistant :\\[0.2cm]
{\begin{tabular}{ll}
\textsc{Schmidt} Johannes & johannes.schmidt@liu.se \\
\end{tabular}}
\\[2cm]

%	DATE SECTION
\vfill
{\normalsize \today} % Date, change the \today to a set date if you want to be precise

\newpage

\end{titlepage}

%--------------------------------------------------------------------------
%	TABLE OF CONTENTS
%--------------------------------------------------------------------------

\pagenumbering{gobble}
\clearpage
\thispagestyle{empty}
\tableofcontents
\clearpage
\pagenumbering{arabic}

%--------------------------------------------------------------------------
%	CONTENT
%--------------------------------------------------------------------------

\section{Introduction}

% Talk about HTTP, TCP socket programming

% Proxy: what it is

% We chose C++

\section{Manual}
\subsection{How to use it}

% compile with make
% configure: browser
% use with terminal instruction

\subsection{How it works}

% req 2: server with CTCPBuffer, then gives it to client
% req 3: server side when parsing url
% req 6: ???
% req 7: command line argument
% req 8: client side, looks for Content-type:

% High level algorithm + details
% Simple drawing of prep. question 3
% Data transfer via  recv and send
% Connection close and keep alive, why
% RFC standard for the use of proxies

\section{Testing}
\subsection{Several cases}
\paragraph{A simple text file}
\paragraph{A simple HTML file}
\paragraph{An HTML file with a bad name}
\paragraph{An HTML file with a good name but bad content}
\paragraph{A valid Google request}
% others
\paragraph{A Google request with bad content}
\paragraph{A YouTube video}
\paragraph{A tour through Wikipedia}
\subsection{Summary of the available services} 

\section{Conclusion}
% possible improvements

    
\end{document}